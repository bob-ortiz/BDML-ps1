% Options for packages loaded elsewhere
\PassOptionsToPackage{unicode}{hyperref}
\PassOptionsToPackage{hyphens}{url}
\PassOptionsToPackage{dvipsnames,svgnames,x11names}{xcolor}
%
\documentclass[
]{article}
\usepackage{amsmath,amssymb}
\usepackage{lmodern}
\usepackage{iftex}
\ifPDFTeX
  \usepackage[T1]{fontenc}
  \usepackage[utf8]{inputenc}
  \usepackage{textcomp} % provide euro and other symbols
\else % if luatex or xetex
  \usepackage{unicode-math}
  \defaultfontfeatures{Scale=MatchLowercase}
  \defaultfontfeatures[\rmfamily]{Ligatures=TeX,Scale=1}
\fi
% Use upquote if available, for straight quotes in verbatim environments
\IfFileExists{upquote.sty}{\usepackage{upquote}}{}
\IfFileExists{microtype.sty}{% use microtype if available
  \usepackage[]{microtype}
  \UseMicrotypeSet[protrusion]{basicmath} % disable protrusion for tt fonts
}{}
\makeatletter
\@ifundefined{KOMAClassName}{% if non-KOMA class
  \IfFileExists{parskip.sty}{%
    \usepackage{parskip}
  }{% else
    \setlength{\parindent}{0pt}
    \setlength{\parskip}{6pt plus 2pt minus 1pt}}
}{% if KOMA class
  \KOMAoptions{parskip=half}}
\makeatother
\usepackage{xcolor}
\usepackage[margin=1in]{geometry}
\usepackage{color}
\usepackage{fancyvrb}
\newcommand{\VerbBar}{|}
\newcommand{\VERB}{\Verb[commandchars=\\\{\}]}
\DefineVerbatimEnvironment{Highlighting}{Verbatim}{commandchars=\\\{\}}
% Add ',fontsize=\small' for more characters per line
\usepackage{framed}
\definecolor{shadecolor}{RGB}{248,248,248}
\newenvironment{Shaded}{\begin{snugshade}}{\end{snugshade}}
\newcommand{\AlertTok}[1]{\textcolor[rgb]{0.94,0.16,0.16}{#1}}
\newcommand{\AnnotationTok}[1]{\textcolor[rgb]{0.56,0.35,0.01}{\textbf{\textit{#1}}}}
\newcommand{\AttributeTok}[1]{\textcolor[rgb]{0.77,0.63,0.00}{#1}}
\newcommand{\BaseNTok}[1]{\textcolor[rgb]{0.00,0.00,0.81}{#1}}
\newcommand{\BuiltInTok}[1]{#1}
\newcommand{\CharTok}[1]{\textcolor[rgb]{0.31,0.60,0.02}{#1}}
\newcommand{\CommentTok}[1]{\textcolor[rgb]{0.56,0.35,0.01}{\textit{#1}}}
\newcommand{\CommentVarTok}[1]{\textcolor[rgb]{0.56,0.35,0.01}{\textbf{\textit{#1}}}}
\newcommand{\ConstantTok}[1]{\textcolor[rgb]{0.00,0.00,0.00}{#1}}
\newcommand{\ControlFlowTok}[1]{\textcolor[rgb]{0.13,0.29,0.53}{\textbf{#1}}}
\newcommand{\DataTypeTok}[1]{\textcolor[rgb]{0.13,0.29,0.53}{#1}}
\newcommand{\DecValTok}[1]{\textcolor[rgb]{0.00,0.00,0.81}{#1}}
\newcommand{\DocumentationTok}[1]{\textcolor[rgb]{0.56,0.35,0.01}{\textbf{\textit{#1}}}}
\newcommand{\ErrorTok}[1]{\textcolor[rgb]{0.64,0.00,0.00}{\textbf{#1}}}
\newcommand{\ExtensionTok}[1]{#1}
\newcommand{\FloatTok}[1]{\textcolor[rgb]{0.00,0.00,0.81}{#1}}
\newcommand{\FunctionTok}[1]{\textcolor[rgb]{0.00,0.00,0.00}{#1}}
\newcommand{\ImportTok}[1]{#1}
\newcommand{\InformationTok}[1]{\textcolor[rgb]{0.56,0.35,0.01}{\textbf{\textit{#1}}}}
\newcommand{\KeywordTok}[1]{\textcolor[rgb]{0.13,0.29,0.53}{\textbf{#1}}}
\newcommand{\NormalTok}[1]{#1}
\newcommand{\OperatorTok}[1]{\textcolor[rgb]{0.81,0.36,0.00}{\textbf{#1}}}
\newcommand{\OtherTok}[1]{\textcolor[rgb]{0.56,0.35,0.01}{#1}}
\newcommand{\PreprocessorTok}[1]{\textcolor[rgb]{0.56,0.35,0.01}{\textit{#1}}}
\newcommand{\RegionMarkerTok}[1]{#1}
\newcommand{\SpecialCharTok}[1]{\textcolor[rgb]{0.00,0.00,0.00}{#1}}
\newcommand{\SpecialStringTok}[1]{\textcolor[rgb]{0.31,0.60,0.02}{#1}}
\newcommand{\StringTok}[1]{\textcolor[rgb]{0.31,0.60,0.02}{#1}}
\newcommand{\VariableTok}[1]{\textcolor[rgb]{0.00,0.00,0.00}{#1}}
\newcommand{\VerbatimStringTok}[1]{\textcolor[rgb]{0.31,0.60,0.02}{#1}}
\newcommand{\WarningTok}[1]{\textcolor[rgb]{0.56,0.35,0.01}{\textbf{\textit{#1}}}}
\usepackage{graphicx}
\makeatletter
\def\maxwidth{\ifdim\Gin@nat@width>\linewidth\linewidth\else\Gin@nat@width\fi}
\def\maxheight{\ifdim\Gin@nat@height>\textheight\textheight\else\Gin@nat@height\fi}
\makeatother
% Scale images if necessary, so that they will not overflow the page
% margins by default, and it is still possible to overwrite the defaults
% using explicit options in \includegraphics[width, height, ...]{}
\setkeys{Gin}{width=\maxwidth,height=\maxheight,keepaspectratio}
% Set default figure placement to htbp
\makeatletter
\def\fps@figure{htbp}
\makeatother
\setlength{\emergencystretch}{3em} % prevent overfull lines
\providecommand{\tightlist}{%
  \setlength{\itemsep}{0pt}\setlength{\parskip}{0pt}}
\setcounter{secnumdepth}{-\maxdimen} % remove section numbering
\usepackage{booktabs}
\usepackage{longtable}
\usepackage{array}
\usepackage{multirow}
\usepackage{wrapfig}
\usepackage{float}
\usepackage{colortbl}
\usepackage{pdflscape}
\usepackage{tabu}
\usepackage{threeparttable}
\usepackage{threeparttablex}
\usepackage[normalem]{ulem}
\usepackage{makecell}
\usepackage{xcolor}
\ifLuaTeX
  \usepackage{selnolig}  % disable illegal ligatures
\fi
\IfFileExists{bookmark.sty}{\usepackage{bookmark}}{\usepackage{hyperref}}
\IfFileExists{xurl.sty}{\usepackage{xurl}}{} % add URL line breaks if available
\urlstyle{same} % disable monospaced font for URLs
\hypersetup{
  pdftitle={Problem Set 1},
  pdfauthor={Gustavo A. Castillo Alvarez (201812166)},
  colorlinks=true,
  linkcolor={red},
  filecolor={Maroon},
  citecolor={Blue},
  urlcolor={blue},
  pdfcreator={LaTeX via pandoc}}

\title{Problem Set 1}
\author{Gustavo A. Castillo Alvarez (201812166)}
\date{}

\begin{document}
\maketitle

\hypertarget{preparar-espacio-de-trabajo}{%
\section{Preparar espacio de
trabajo}\label{preparar-espacio-de-trabajo}}

Los documentos necesarios para replicar los datos aquí presente se debe
acceder al repositorio \href{https://github.com/bob-ortiz/BDML-ps1}{en
este enlace}.

Primero preparamos el espacio de trabajo y cargamos las librerías a
utilizar. Cargar las librerías necesarias mediante el paquete
\texttt{groundhog} asegura la replicabilidad del código a a pesar de que
ocurran cambios en las dependencias o en las librerías.

\begin{verbatim}
## Installing package into '/Users/upar/Library/R/arm64/4.2/library'
## (as 'lib' is unspecified)
\end{verbatim}

\begin{verbatim}
## Loaded 'groundhog' (version:2.0.1) using R-4.2.1
\end{verbatim}

\begin{verbatim}
## Tips and troubleshooting: https://groundhogR.com
\end{verbatim}

\begin{Shaded}
\begin{Highlighting}[]
\FunctionTok{groundhog.library}\NormalTok{(}\AttributeTok{pkg =}\NormalTok{ pkgs, }\AttributeTok{date =} \StringTok{"2022{-}08{-}31"}\NormalTok{)}
\end{Highlighting}
\end{Shaded}

\begin{verbatim}
## Succesfully attached 'rvest_1.0.3'
\end{verbatim}

\begin{verbatim}
## 
## Attaching package: 'dplyr'
\end{verbatim}

\begin{verbatim}
## The following objects are masked from 'package:stats':
## 
##     filter, lag
\end{verbatim}

\begin{verbatim}
## The following objects are masked from 'package:base':
## 
##     intersect, setdiff, setequal, union
\end{verbatim}

\begin{verbatim}
## Succesfully attached 'dplyr_1.0.9'
\end{verbatim}

\begin{verbatim}
## Loading required package: ggplot2
\end{verbatim}

\begin{verbatim}
## Loading required package: lattice
\end{verbatim}

\begin{verbatim}
## Succesfully attached 'caret_6.0-93'
\end{verbatim}

\begin{verbatim}
## Warning in !is.null(rmarkdown::metadata$output) && rmarkdown::metadata$output
## %in% : 'length(x) = 3 > 1' in coercion to 'logical(1)'
\end{verbatim}

\begin{verbatim}
## 
## Attaching package: 'kableExtra'
\end{verbatim}

\begin{verbatim}
## The following object is masked from 'package:dplyr':
## 
##     group_rows
\end{verbatim}

\begin{verbatim}
## Succesfully attached 'kableExtra_1.3.4'
\end{verbatim}

\begin{verbatim}
## Registered S3 methods overwritten by 'readr':
##   method                    from 
##   as.data.frame.spec_tbl_df vroom
##   as_tibble.spec_tbl_df     vroom
##   format.col_spec           vroom
##   print.col_spec            vroom
##   print.collector           vroom
##   print.date_names          vroom
##   print.locale              vroom
##   str.col_spec              vroom
\end{verbatim}

\begin{verbatim}
## -- Attaching packages --------------------------------------- tidyverse 1.3.2 --
## v tibble  3.1.8     v purrr   0.3.4
## v tidyr   1.2.0     v stringr 1.4.1
## v readr   2.1.2     v forcats 0.5.2
## -- Conflicts ------------------------------------------ tidyverse_conflicts() --
## x dplyr::filter()          masks stats::filter()
## x kableExtra::group_rows() masks dplyr::group_rows()
## x readr::guess_encoding()  masks rvest::guess_encoding()
## x dplyr::lag()             masks stats::lag()
## x purrr::lift()            masks caret::lift()
## Succesfully attached 'tidyverse_1.3.2'
## 
## 
## Please cite as: 
## 
## 
##  Hlavac, Marek (2022). stargazer: Well-Formatted Regression and Summary Statistics Tables.
## 
##  R package version 5.2.3. https://CRAN.R-project.org/package=stargazer 
## 
## 
## Succesfully attached 'stargazer_5.2.3'
\end{verbatim}

\hypertarget{datos}{%
\section{Datos}\label{datos}}

\hypertarget{adquisiciuxf3n-de-datos}{%
\subsection{Adquisición de datos}\label{adquisiciuxf3n-de-datos}}

Para obtener los datos de la base GEIH realicé el raspado web a partir
de la
\href{https://ignaciomsarmiento.github.io/GEIH2018\%20sample/}{página
web del profesor}. Para obtener la base completa primero corro el script
\texttt{1-scraping.R}. Para la

\hypertarget{punto-2-age-earnings-profile}{%
\section{Punto 2: Age-earnings
profile}\label{punto-2-age-earnings-profile}}

\begin{Shaded}
\begin{Highlighting}[]
\NormalTok{df }\OtherTok{\textless{}{-}} \FunctionTok{read.csv}\NormalTok{(}\StringTok{"../stores/encuesta.csv"}\NormalTok{)}
\NormalTok{df }\OtherTok{\textless{}{-}}\NormalTok{ df }\SpecialCharTok{\%\textgreater{}\%} 
  \FunctionTok{mutate}\NormalTok{(}\AttributeTok{age\_sqr =}\NormalTok{ age }\SpecialCharTok{*}\NormalTok{ age)}

\NormalTok{model2 }\OtherTok{\textless{}{-}} \FunctionTok{lm}\NormalTok{(df}\SpecialCharTok{$}\NormalTok{ingtotob}\SpecialCharTok{\textasciitilde{}}\NormalTok{df}\SpecialCharTok{$}\NormalTok{age}\SpecialCharTok{+}\NormalTok{df}\SpecialCharTok{$}\NormalTok{age\_sqr)}
\CommentTok{\#shortgeih \textless{}{-} geih \%\textgreater{}\% filter(age\textgreater{}=18, dsi==0)}
\CommentTok{\#model3 \textless{}{-} lm(shortgeih$ingtot\textasciitilde{}shortgeih$age+(shortgeih \%\textgreater{}\% mutate(age2 = age*age))$age2)}
\FunctionTok{stargazer}\NormalTok{(model2, }\AttributeTok{type =} \StringTok{"text"}\NormalTok{)}
\end{Highlighting}
\end{Shaded}

\begin{verbatim}
## 
## ===============================================
##                         Dependent variable:    
##                     ---------------------------
##                              ingtotob          
## -----------------------------------------------
## age                       100,809.700***       
##                             (3,185.367)        
##                                                
## age_sqr                     -972.023***        
##                              (34.954)          
##                                                
## Constant                 -1,041,893.000***     
##                            (64,160.770)        
##                                                
## -----------------------------------------------
## Observations                  27,289           
## R2                             0.042           
## Adjusted R2                    0.042           
## Residual Std. Error 2,131,051.000 (df = 27286) 
## F Statistic         597.356*** (df = 2; 27286) 
## ===============================================
## Note:               *p<0.1; **p<0.05; ***p<0.01
\end{verbatim}

\hypertarget{punto-3-the-gender-earnings-gap}{%
\section{Punto 3: The gender earnings
gap}\label{punto-3-the-gender-earnings-gap}}

\[
\log(ingtotob)=\beta_1+\beta_2 sex+u
\]

\begin{Shaded}
\begin{Highlighting}[]
\NormalTok{df}\SpecialCharTok{$}\NormalTok{female }\OtherTok{\textless{}{-}} \FunctionTok{ifelse}\NormalTok{(df}\SpecialCharTok{$}\NormalTok{sex}\SpecialCharTok{==}\DecValTok{0}\NormalTok{, }\DecValTok{1}\NormalTok{, }\DecValTok{0}\NormalTok{)}
\NormalTok{df }\OtherTok{\textless{}{-}}\NormalTok{ df }\SpecialCharTok{\%\textgreater{}\%}
  \FunctionTok{mutate}\NormalTok{(}\AttributeTok{l\_y=}\FunctionTok{log}\NormalTok{(ingtotob))}
\end{Highlighting}
\end{Shaded}

\begin{Shaded}
\begin{Highlighting}[]
\CommentTok{\# Análisis descriptivo}
\DocumentationTok{\#\# Y}
\CommentTok{\#class(df$ingtotob)}
\CommentTok{\#plot(hist(df$ingtotob))}

\FunctionTok{ggplot}\NormalTok{(}\AttributeTok{data =}\NormalTok{ df) }\SpecialCharTok{+} 
  \FunctionTok{geom\_smooth}\NormalTok{(}\AttributeTok{mapping =} \FunctionTok{aes}\NormalTok{(}\AttributeTok{x=}\NormalTok{age, }\AttributeTok{y=}\NormalTok{ingtotob, }\AttributeTok{linetype =} \FunctionTok{as.factor}\NormalTok{(maxEducLevel))) }
\end{Highlighting}
\end{Shaded}

\begin{verbatim}
## `geom_smooth()` using method = 'gam' and formula 'y ~ s(x, bs = "cs")'
\end{verbatim}

\begin{verbatim}
## Warning: Removed 4888 rows containing non-finite values (stat_smooth).
\end{verbatim}

\begin{verbatim}
## Warning: Computation failed in `stat_smooth()`:
## x has insufficient unique values to support 10 knots: reduce k.
\end{verbatim}

\includegraphics{Problem_Set_1-Gustavo_files/figure-latex/unnamed-chunk-5-1.pdf}

\begin{Shaded}
\begin{Highlighting}[]
\CommentTok{\#ggplot(df)+}
\CommentTok{\#  geom\_boxplot(aes(as.factor(maxEducLevel), ingtotob))}
  


\DocumentationTok{\#\# X}
\CommentTok{\#x \textless{}{-} c(df$age, df$p6210, df$dsi, df$maxEducLevel, df$estrato1)}
  
\CommentTok{\# Age{-}earnings profile}
\end{Highlighting}
\end{Shaded}


\end{document}
